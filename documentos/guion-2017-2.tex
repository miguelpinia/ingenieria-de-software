\documentclass[11pt]{article}
\usepackage[utf8]{inputenc}
\usepackage[spanish]{babel}
\usepackage{xtab, paralist}
\usepackage{amsmath}
\usepackage{amsfonts}
\usepackage{amssymb}
\usepackage{makeidx}
\usepackage{graphicx}
\usepackage{verbatim}
\usepackage[left=2cm,right=2cm,top=2cm,bottom=2cm]{geometry}
\graphicspath{ {./figures/} }
\author{Profesor: Hanna Oktaba\\Ayudante: Ósale Dirán\\Ayudante de laboratorio: Miguel Piña}
\date{\today}
\title{Ingeniería de Software }
\begin{document}
\maketitle

\section*{Guión del laboratorio}

% | semana | clases de laboratorio | entregas de los alumnos |

\tablefirsthead{
  \hline
  \textbf{Semana} & \textbf{Clase de laboratorio} & \textbf{Entregas
  de los alumnos} \\
}
\tablehead{
  \hline
  \multicolumn{3}{|r|}{\small\sl Continúa desde la página previa}\\
  \hline
  \textbf{Semana} & \textbf{Clase de laboratorio} & \textbf{Entregas
  de los alumnos} \\
}
\tabletail{
  \hline
  \multicolumn{3}{|r|}{\small\sl Continúa en la siguiente página}\\
  \hline
}
\tablelasttail{
  \hline
}
\begin{center}
  \begin{mpxtabular}{|l|p{0.53\textwidth}|p{0.27\textwidth}|}
    \hline
    1 & No hay clase & - \\
    31 de enero & & \\
    \hline
    2 & Presentación, introducción a las tecnologías usadas para el
    proyecto \begin{compactitem}
      \item  JEE
      \item Entorno de desarrollo (Netbeans)
      \item Git
      \item PostgreSQL
    \end{compactitem} & - \\
    7 de febrero & & \\
    \hline
    3 & Introducción formal a Java EE, desarrollo de una primera
    aplicación web (Hello World) & - \\
    14 de febrero & & \\
    \hline
    4 & Conceptos claves de JEE
    \begin{compactitem}
      \item JSP
      \item Servlets
    \end{compactitem}
    & \\
    21 de febrero & & \\
    \hline
    5 & Introducción formal a git
    \begin{compactitem}
      \item Comando básicos
      \item Creación de un repositorio de software
      \end{compactitem} & Práctica 1\\
    28 de febrero & & \\
    \hline
    6 & Introducción a PostgreSQL & - \\
    7 de marzo & & \\
    \hline
    7 & Conceptos de MVC y su aplicación con JEE y postgreSQL     \begin{compactitem}
    \item JPA (Java Persistence API)
    \item JUnit
    \end{compactitem} &
    Práctica 2 \\
    14 de marzo & & \\
    \hline
    8 & Construcción \begin{compactitem}
    \item Base de datos
    \item Modelo de la aplicación
    \end{compactitem}
    & - \\
    21 de marzo & & \\
    \hline
    9 & Construcción \begin{compactitem}
    \item Controles
    \item Vista
    \end{compactitem}
    & - \\
    28 de marzo & & \\
    \hline
    10 & Integración y Pruebas Unitarias (JUnit) & - \\
    4 de abril & & \\
    \hline
    11 & Revisión de código construido & - \\
    18 de abril & & \\
    \hline
    12 & Retrospectiva de iteración & - \\
    25 de abril & & \\
    \hline
    13 & Construcción de segundo ciclo \begin{compactitem}
    \item Construcción de los casos de uso para el segundo ciclo.
    \end{compactitem} & - \\
    2 de mayo & & \\
    \hline
    14 & Buenas prácticas de desarrollo \begin{compactitem}
    \item Introducción a checkstyle
    \end{compactitem} & - \\
    9 de mayo & & \\
    \hline
    15 & Revisión del código de segunda iteración & - \\
    16 de mayo & & \\
    \hline
    16 & Retrospectiva del curso & Examen de retrospección  \\
  \end{mpxtabular}
\end{center}
\end{document}
